\documentclass{article}

\usepackage{amsmath}
\usepackage{amsfonts}

\begin{document}
The motivation of the study of signals and systems is to analyze properties that change over time. 
For example, from introductory physics we know that the speed of an object and force it feels will change with time when attached to a perturbed spring.
Later in this chapter, when we mathematically define what a signal is, stop and ask yourself whether our definition captures our intent to describe properties such as these.

In addition to signals themselves we may be interested in how one signal affects another signal. 
In the spring for example, both its restoring force and its length will change over time.
With knowledge of both the restoring force and the length at any given time $t$ can we determine the relationship between these signals?
To analyze how signals affect each other, we define a system as a mathematical object that transforms signals into signals. 
As an analogue, when we apply a function to a number we get another number and therefore we can say a function is a mathematical object that transforms numbers into numbers.
In a similar vein, when we apply a system to a signal we get another signal. 
To analyze the relationship between two signals, we can view one as the input to a system and the other as an output.

To mathematically define what a signal is, I will first present three definitions that capture a portion of our intent above. 
After this, I'll show that these three definitions can be reduced into just one definition. 

As mentioned above, we know that the speed of an object and the force it experiences will change with time when attached to a spring. 
An important subtlety not entirely captured above is whether we are measuring these quantities or modelling them.
If we are modelling a property, it may make sense to specify its value at one particular time and then predict its value for the rest of eternity after that point as it remains unperturbed.
In the world of measurement however, any period of measurement must have a beginning and an end and it would make little sense to allow a signal to take a value for any time $t$.

To handle the first case, we define a signal as a function mapping the interval $[0,\infty)$ to a set of complex numbers. 
Why complex numbers?
In this case, the complex numbers have the ability to capture all of our intention by restricting ourselves to $\mathbb{R} \subset \mathbb{C}$ while allowing us greater flexability.  

\end{document}
