\documentclass{article}

\begin{document}
The motivation of the study of signals and systems is to analyze properties that change over time. 
For example, a spring may bounce up and down and we can observe how its vertical position changes over time. 
In more mathematical language, at any given time $t$, we can measure the position of the spring and call this position $x(t)$.
We call these properties which change over time, signals.

In addition to signals themselves we may be interested in how one signal affects another signal. 
In the spring for example, both its restoring force and its length will change over time.
With knowledge of both the restoring force and the length at any given time $t$ can we determine the relationship between these signals?
To analyze how signals affect each other, we define a system as a mathematical object that transforms signals into signals. 
As an analogue, when we apply a function to a number we get another number and therefore we can say a function is a mathematical object that transforms numbers into numbers.
In a similar vein, when we apply a system to a signal we get another signal. 
To analyze the relationship between two signals, we can view one as the input to a system and the other as an output.

Before we get into a more mathematical definition of what we've explored thus far, let us take a moment to examine some of the subtleties of the above discussion. 
Firstly, a signal was defined as any property where given a time $t$, we can measure the value of the property $x(t)$.
What are appropriate values of $t$?
In practice, we allow $t$ to take any real value from $-\infty$ to $\infty$.
\end{document}
