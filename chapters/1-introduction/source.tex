\documentclass{article}

\begin{document}
    This textbooks aims to provide a solid mathematical framework for the study of signals and systems.
    The techniques illustrated in this book form the basis for many electrical engineering techniques such as measurement, sampling and filtering.
    There are a few key themes that the reader should keep in mind throughout this course.

    Firstly, transforms such as the Fourier series or the Laplace transform are invaluable tools that allow electrical engineers to solve problems more easily by transforming them into problems more easily solved.
    Although the definitions may seem arbitrary at first, these formula were created with this intention in mind.
    Secondly, the mathematics of signals and systems allow engineers to create representations of analog, continuous time phenomena in such a way that they can be stored as bits in a computer, manipulated and transformed back into the analog world. 
    With an understanding of signals and systems we can understand what assumptions are made when a set of analog measurements are stored on a computer and when it is possible for the original phenomena being measured to be reconstructed.
\end{document}
